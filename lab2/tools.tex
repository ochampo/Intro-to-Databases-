
\documentclass{report}
\usepackage{a4wide}
\title{\emph{ Lab 2  }}
\author{Daniel Ocampo}
\begin{document}
\maketitle 
\linespread{1.0}\selectfont




 Questions 

\begin{enumerate}
\item What is the relation schema for each data set that describes the association between their
entities?

\item How are the column headers helpful for designing your schemas?

The column help decide what tables I would need to make, kind of like my database architecture. 

\item What is an appropriate primary key for each data set?

The primary key should be entry because there i Think the data is one of a kind. 

\item What kinds of memory (VARCHAR memory allocation) do you think you need to create
your base? Why?

I think we need a not null because we need an input for all of these data because we are reading it through a file. 


\end{enumerate}




Part 2
\begin{enumerate}




\item Write a query that will return the count of elements in the Entry columns in both tables.

SELECT(SELECT count(entry) from Apop) + (SELECT count(entry) from Park);

\item Write a query that will return the distinct count elements in the Entry column.
select count(distinct(Entry)) from Apop;
\item Discuss: From the above two queries, is this column a good primary key for each table? why or
why not? (if not, then what column would you recommend, instead?)

Yes because everything seems to be unique if it wasnt we would have some problems finding a specific item in the database. 

\item Write a query that will return the number of records associated with ‘‘Zea mays(Maize)’’ in
both tables. You might want to first determine what where entity is found in the table to create
your query.
\begin{itemize}
\item The entity is found in the orginism section and the command that is needed for it  is SELECT count(distinct(Entry)) From Park where Organism == "Zea mays (Maize)";
9
\item SELECT count(distinct(Entry)) From Apop where Organism == "Zea mays (Maize)";
13
\end{itemize}









\item Write a query that will report how many organisms were listed in each table.

\begin{itemize}

\item SELECT(SELECT count(Organism) from Apop) + (SELECT count(organism) from Park);
227450
\item  SELECT count(Organism) from Apop ;
100082
\item  SELECT count(Organism) from Park ;
127368

\end{itemize}
 \item  Write a query that will return the number of organisms which are common to both
tables (as in, the intersection of the tables for this attribute).

\begin{itemize}
\item SELECT count (*) FROM ( SELECT Organism from Apop INTERSECT SELECT Organism FROM Park);

147		
\end{itemize}
\item Write a query that will return the number of proteins which are common to Apoptosis
and Parkinson’s, which are associated to the Zea mays (Maize) organism.

So what was my orginal Idea was to find what both of them had common with zea mays because for some reason I thought the protein had zea mays. When in fact the organsinms had zea mays. It must of been my lack of biology.  



SELECT count (*) FROM ( SELECT Proteinname from Apop where Proteinname== "Zea mays"
INTERSECT SELECT Organism FROM Park where Proteinname=="Zea mays");\\
4 \\
After getting help I discoverd that we can use random variables that used to distinguse tables in a way.
For example in the bottom SQL command with a variable it can be any variable at that. I
feel like we declare it after wich is why I was A little confused about it. So we are kind of like setting Apop to a and Park to p. Now we can use it to distinguish between the two tables. For comparison reasons we use the random variable because it gives us the ability to compare. 

select count(distinct(a.Proteinname)) from Apop a, Park p \\ 
where a.Proteinname == p.Proteinname and a.organism = "Zea mays (Maize)";


\item Modify this query to print the first 15 proteins (Entry entites), according to
Zea mays (Maize) are related.\\
\textless sqlite$>$  Select a.Entry, p.Entry from Apop a,Park p where a.Organism ="Zea mays (Maize)" LIMIT 15;
A0A096QZ88|A0A061IEP5 \\
A0A096TM17|A0A061IEP5 \\
B6STV7|A0A061IEP5  \\
B6SUW1|A0A061IEP5  \\
B6T3A3|A0A061IEP5  \\
B6T3H2|A0A061IEP5  \\
B6T9F6|A0A061IEP5  \\
B6TB21|A0A061IEP5  \\
B6TZ68|A0A061IEP5  \\
C4J813|A0A061IEP5   \\
O81214|A0A061IEP5  \\
Q6PLR8|A0A061IEP5  \\
Q6PLR9|A0A061IEP5 \\
A0A096QZ88|A0A067RA11 \\
A0A096TM17|A0A067RA11  \\
$<$sqlite$>$  Select a.Entry, p.Entry from Apop a,Park p where p.Organism ="Zea mays (Maize)" LIMIT 15;
A0A009GCL8|C0JR65 \\
A0A009GCL8|C0JR66  \\
A0A009GCL8|C0JR67  \\
A0A009GCL8|C0JR68  \\
A0A009GCL8|C0JR69   \\
A0A009GCL8|I6WV73   \\
A0A009GCL8|I6XGM5   \\
A0A009GCL8|K7S1X3  \\
A0A009GCL8|Q9T6S0  \\
A0A009GLJ9|C0JR65  \\
A0A009GLJ9|C0JR66 \\
A0A009GLJ9|C0JR67 \\
A0A009GLJ9|C0JR68 \\
A0A009GLJ9|C0JR69 \\
A0A009GLJ9|I6WV73 \\
\item  Create a query to determine how many genes are in common in both tables, for all
organisms (i.e., the intersection of all information about genes across all organisms).

 select count( ) from Apop a, Park p where a.gene\_names ==P.gene\_names and a.Organism == p.Organism;     \\
 
11673


\item Create another query to determine the names of first ten of these genes which
are at the intersection of both tables, across all organisms.

select a.gene\_names,p.gene\_names,a.Organism,p.Organism from Park p , Apop a where p.gene\_names = a.gene\_names limit 15 ;




PARK2 PRKN|PARK2 PRKN|Homo sapiens (Human)|Homo sapiens (Human)\\
LRRK2 PARK8|LRRK2 PARK8|Homo sapiens (Human)|Homo sapiens (Human)\\
MAPT MAPTL MTBT1 TAU|MAPT MAPTL MTBT1 TAU|Homo sapiens (Human)|Homo sapiens (Human)\\
SQSTM1 ORCA OSIL|SQSTM1 ORCA OSIL|Homo sapiens (Human)|Homo sapiens (Human)\\
HTRA2 OMI PRSS25|HTRA2 OMI PRSS25|Homo sapiens (Human)|Homo sapiens (Human)\\
AIMP2 JTV1 PRO0992|AIMP2 JTV1 PRO0992|Homo sapiens (Human)|Homo sapiens (Human)\\
Btk Bpk|Btk Bpk|Mus musculus (Mouse)|Mus musculus (Mouse)\\
CDK5 CDKN5|CDK5 CDKN5|Homo sapiens (Human)|Homo sapiens (Human)\\
Ddit4 Dig2 Redd1 Rtp801|Ddit4 Dig2 Redd1 Rtp801|Mus musculus (Mouse)|Mus musculus (Mouse)\\
Ddit4 Redd1 Rtp801|Ddit4 Redd1 Rtp801|Rattus norvegicus (Rat)|Rattus norvegicus (Rat)\\
DDIT4 REDD1 RTP801|DDIT4 REDD1 RTP801|Homo sapiens (Human)|Homo sapiens (Human)\\
Bm1\_48140|Bm1\_48140|Brugia malayi (Filarial nematode worm)|Brugia malayi (Filarial nematode worm)\\
Rattus norvegicus (Rat)|Rattus norvegicus (Rat)\\
Rattus norvegicus (Rat)|Rattus norvegicus (Rat)\\
Rattus norvegicus (Rat)|Rattus norvegicus (Rat)\\



\end{enumerate}

\end{document}



